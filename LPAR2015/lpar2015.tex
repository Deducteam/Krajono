% LLNCStmpl.tex
% Template file to use for LLNCS papers prepared in LaTeX
%websites for more information: http://www.springer.com
%http://www.springer.com/lncs

\documentclass{llncs}
%Use this line instead if you want to use running heads (i.e. headers on each page):
%\documentclass[runningheads]{llncs}
\usepackage{amssymb}
\usepackage{amsmath}
\usepackage{graphicx}
\usepackage{url}
\usepackage{color}
\usepackage{float}
\usepackage{tabularx}
\usepackage{multirow}
\setlength{\extrarowheight}{1pt}
\usepackage{fancyvrb}
\usepackage{hyperref}
\hypersetup{plainpages=false,pdfborderstyle={/S/U/W 0}}
\hypersetup{colorlinks=true,linkcolor=blue,citecolor=blue}
\floatstyle{plain} \newfloat{example}{thp}{lop} \floatname{example}{\textbf{Example}}
\newtheorem{mydef}{Definition}
\newtheorem{myprop}{Property}
\begin{document}
\title{ELPI: fast, Embeddable, \lp{} Interpreter}

%If you're using runningheads you can add an abreviated title for the running head on odd pages using the following
%\titlerunning{abreviated title goes here}
%and an alternative title for the table of contents:
%\toctitle{table of contents title}

%\subtitle{Subtitle Goes Here}

%For a single author
%\author{Author Name}

%For multiple authors:

\author{Cvetan~Dunchev,$^1$~Ferruccio~Guidi,$^1$~Claudio~Sacerdoti~Coen,$^1$~Enrico~Tassi$^2$}


%If using runnningheads you can abbreviate the author name on even pages:
%\authorrunning{abbreviated author name}
%and you can change the author name in the table of contents
%\tocauthor{enhanced author name}

%For a single institute
\institute{Department of Computer Science,
University of Bologna,~%\\ Mura Anteo Zamboni 7, 40127 Bologna, Italy \\
\email{name.surname@unibo.it}
\and Inria Sophia-Antipolis,~%\\ 2004 route des Lucioles - BP 93, 06902 Sophia Antipolis Cedex, France
\email{name.surname@inria.fr}}

% If authors are from different institutes
%\institute{First Institute Name \email{email address} \and Second Institute Name\thanks{Thank you to...} \email{email address}}

%to remove your email just remove '\email{email address}'
% you can also remove the thanks footnote by removing '\thanks{Thank you to...}'

\newcommand{\frag}{Reduction-Free Fragment}
\newcommand{\lp}{$\lambda$Prolog}
\newcommand{\Ll}{\ensuremath{\mathcal{L}_\lambda}}
\newcommand{\rff}{\ensuremath{\mathcal{L}_\lambda^\beta}}
\newcommand{\elpi}{ELPI}
\newcommand{\tedius}{Teyjus}
\newcommand{\CSC}[1]{\textcolor{red}{#1}}
\newcommand{\FG}[1]{\textcolor{magenta}{#1}}

\maketitle

\begin{abstract}
We present a new interpreter for \lp{} that runs consistently faster than
the byte code compiled by \tedius{}, that is believed to be the best
available implementation for \lp. 
The key insight is the identification of a fragment of
the language, which we call \frag{}, that occurs quite naturally in \lp{}
programs and that admits constant time reduction and unification rules.
% The implementation exploits De Bruijn levels and no explicit substitutions,
% whereas \tedius{} is based on De Bruijn indexes and explicit substitutions
% (the suspension calculus).
\end{abstract}

\section{Introduction}% and State of the Art}
\lp{} is a logic programming language based on an intuitionistic fragment of
Church's Simple Theory of Types. An extensive introduction to the language
with examples can be found in~\cite{Miller:2012:PHL:2331097}. \tedius{}
\cite{DBLP:conf/cade/NadathurM99,DBLP:journals/corr/abs-0911-5203} is a
compiler for \lp{} %implemented by Gopalan Nadathur and others
that is considered to be the fastest
implementation of the language. 
%Previous slower implementations are described
%in~\cite{}. \CSC{OTHER IMPLEMENTATIONS IN ISABELLE ETC. MISSING}
The main difference with respect to Prolog is that \lp{} manipulates
$\lambda$-tree expressions, i.e. syntax containing binders. Therefore, the
natural application of \lp{} is meta-programming (see~\cite{LPAZ} for
an interesting discussion), including: automatic generation of programs from
specifications; animation of operational semantics;
program transformations and implementation of type checking algorithms.

Via the Curry-Howard isomorphism, a type-checker is a proof-checker, the main
component of an interactive theorem prover (ITP). Indeed the motivation of our
interest in \lp{} is that we are looking for the best 
language to implement the so called \emph{elaborator} component of an ITP.
The elaborator is used to type check the terms input by
the user. Such data, for conciseness reasons, is typically incomplete and
the ITP is expected to infer what is missing. The possibility to
extend Coq's built-in elaborator with user provided ``logic programs'' (in the
form of type classes or unification hints) to infer the missing pieces of
information turned out to be a key ingredient in successful formalizations
like~\cite{gonthier:hal-00816699}. Embedding a \lp{} interpreter in an ITP
would enable the elaborator and its extensions to be expressed in the same,
high level, language. A crucial requisite for this plan to be realistic is
the efficiency of the \lp{} interpreter.

In this paper we introduce ELPI, a fast \lp{} interpreter written in OCaml
that can be easily embedded in OCaml softwares, like Coq.
In particular we focus on the insight that makes ELPI fast when dealing with
binders by identifying a reduction-free fragment (\rff{}) of \lp{} that, if
implemented correctly, admits constant-time unification and reduction
operations.
We analyze the role of $\beta$-reduction in Section~\ref{sec:beta} and
higher order unification in Section~\ref{sec:ho}; we discuss bound names
representations in Section~\ref{sec:dbl}; we define \rff{} 
in Section~\ref{sec:fragment} and we assess the results in
Section~\ref{sec:benchmarks}.

% We don't mention constraints here, too much vapor ware IMO.
%\lp{} with Constraints (\`a la CLP) is the best choice.

\section{The two roles of $\beta$-reduction in \lp{}}
\label{sec:beta}

\begin{example}[b]
\begin{center}
\begin{tabular}{cc}
\begin{minipage}{4.8cm}
\begin{Verbatim}[numbers=left,numbersep=1pt,frame=leftline]
of (app M N) B :-
  of M (arr A B), of N A.
of (lam F) (arr A B) :-
  pi x\ of x A => of (F x) B.
\end{Verbatim}
\end{minipage}~~
&
~~~\begin{minipage}{6.5cm}
\begin{Verbatim}[numbers=left,firstnumber=5,numbersep=1pt,frame=leftline]
cbn (lam F) (lam F).
cbn (app (lam F) N) M :- cbn (F N) M.
cbn (app M N) R :-
  cbn M (lam F), cbn (app (lam F) N) R.
\end{Verbatim}
\end{minipage}
\end{tabular}
\end{center}
\caption{\label{example1} Type checker and Weak CBN for simply typed $\lambda$-calculus.}
\end{example}

Example~\ref{example1} implements type-checking and
reduction for $\lambda$-terms represented in $\lambda$-tree syntax.
For instance,
$(\lambda x.xx)$ is encoded
as \verb+(lam (x\ app x x))+ where \verb+x\ F+ is
the $\lambda$-abstraction of \lp{}, that binds \verb+x+ in \verb+F+;
\verb+lam+ is the constructor for object-level abstraction, that builds
a term of type $\mathcal{T}$ from a function of type
$\mathcal{T} \to \mathcal{T}$, with $\mathcal{T}$ the type of representations
of $\lambda$-terms; \verb+app+ takes two terms of type $\mathcal{T}$ and builds
their object-level application of type $\mathcal{T}$. Following the tradition of Prolog, capitals letters denote unification variables.

The second clause for the \verb+of+ predicate shows a recurrent pattern in
\lp: in order to analyze an higher order term, one needs to recurse
under a binder. This is achieved
combining the forall quantifier, written \verb+pi x\ G+, with logical
implication \verb+H => I+. The operational semantics implements the
standard introduction rules of implication and the universal quantifier:
the forall quantifier declares a
new local constant \verb+x+, meant to be fresh in the entire program;
logical implication temporarily
augments the program with the new axiom \verb+H+ about \verb+x+.

In the example, the functional (sub-)term
\verb+F+ is applied to the fresh constant \verb+x+. Being \verb+F+ a
function, the $\beta$-redex \verb+(F x)+, once reduced, denotes the body of
our object-level function where the bound variable is replaced by the fresh
constant \verb+x+.
The implication is used to
assume \verb+A+ to be the type of \verb+x+, in order to prove that the body of
the abstraction has type \verb+B+ and therefore the whole abstraction has type
\verb+(arr A B)+ (i.e. $A \to B$). Note that, unlike in
the standard presentation of the typing rules, we do not need to manipulate an
explicit context $\Gamma$ to type the free variables. Instead the assumptions
of the form \verb+(of x A)+ are just added to the program's clauses, and \lp{}
takes care of dropping them when \verb+x+ goes out of scope.
Example: if the initial goal is
\verb+(of (lam (w\ app w w)) T)+ by applying the second clause we assign
\verb+(arr A B)+ to \verb+T+ and generate
a new goal \verb+(of (app c c) B)+ (where \verb+c+ is the fresh constant
substituted for \verb+w+) to be solved with the extra clause \verb+(of c A)+
at disposal.

In the type-checking example, the meta-level $\beta$-reduction is only employed
to inspect a term under a binder by replacing the bound name with a fresh
constant. The reduction example in line 6
shows instead a radically different pattern: in order to implement object-level
substitution --- and thus object-level $\beta$-reduction --- we use the
meta-level $\beta$-reduction. E.g. if \verb+F+ is \verb+(w\ app w w)+
then \verb+(F N)+ reduces to \verb+(app N N)+. Note that in this case
$\beta$-reduction is fully general, because it replaces a name with a
general term, not constrained to be a fresh constant.
This distinction is crucial in the definition of \rff{} in
Section~\ref{sec:fragment}.

\section{Higher Order unification}% and captures}
\label{sec:ho}
Higher order (HO) unification admits no most general unifiers (MGUs), forcing
implementations to enumerate all solutions or delay the flexible-rigid
and the flexible-flexible problems. Moreover, the presence of binders
requires a way to avoid captures, i.e. to check that unification variables
are instantiated with terms containing only bound variable in their scope.

To cope with the absence of MGUs, Dale Miller identified
in~\cite{Miller91alogic} a well-behaved fragment (\Ll{}) of higher-order
unification that admits MGUs and that is stable under \lp{} resolution. 
The restriction defining \Ll{} is that unification variables can
only be applied to (distinct) variables (i.e. not arbitrary terms) that are
not already in the scope of the variable.
Such fragment can effectively serve as a primitive for a
programming language and indeed \tedius{} 2.0 is built around this fragment:
no attempt to enumerate all possible unifiers is performed, and unification
%\marginpar{cite huet? no space IMO}
problems falling outside \Ll{} are just delayed. Many interesting \lp{}
programs can be rewritten to fall in the fragment. For example, we can
make \verb+cbn+ of Example~\ref{example1} stay in \Ll{} by replacing
line 6 (that contains the offending \verb+(F N)+ term) with the following code:
\vspace{-0.4em}
\begin{center}
\small
\begin{minipage}{10cm}
\begin{Verbatim}[numbers=left,numbersep=1pt,frame=leftline]
cbn (app (lam F) N) M :- subst F N B, cbn B M.
subst F N B :- pi x\ copy x N => copy (F x) B.
copy (lam F1) (lam F2) :- pi x\ copy x x => copy (F1 x) (F2 x).
copy (app M1 N1) (app M2 N2) :- copy M1 M2, copy N1 N2.
\end{Verbatim}
\end{minipage}
\end{center}
\vspace{-0.3em}
The idea of \verb+subst+ is that the term \verb+F+ is recursively copied in
the following way: each bound variable is copied in itself but for the top one
that is replaced by \verb+N+.
The interested reader can find an longer discussion about \verb+copy+
in~\cite[page 199]{Miller:2012:PHL:2331097}.
The \verb+of+ program falls naturally in \Ll{}, since \verb+F+ is only applied
the fresh variable \verb+x+ (all unification variables in a \lp{} program are
implicitly existentially bound in front of the clause, so \verb+F+ does not
see \verb+x+). The same holds for \verb+copy+.

In \lp{} unification takes place under
a mixed prefix of $\forall$ and $\exists$ quantifiers; their order
determines if a unification variable (an existential) can be assigned to
(proved by) a term that contain a universally quantified variable.
E.g. $\forall x,\exists Y, Y = x$ is provable while
$\exists Y,\forall x, Y = x$ is not. An implementation can
keep track of the scoping constraints using \emph{levels}.
When a clause is used to match a goal in a context of length
$n$, the universally quantified variables of the clause are instantiated to
unification variables $X^n$ where the level $n$ records that $X$ has only
visibility of the initial prefix of length $n$ of the context.
If later a fresh constant is added by the \verb+pi+ rule, the constant
occupies position $n+1$ (its level is $n+1$) and it will not be allowed to
occur in instances of variables $X^n$.
From now on we will write levels of constants and variables in superscript.

If we run the program
\verb+(of (lam f\lam w\app f w) T+$^0$\verb+)+, after two steps the goal becomes
\verb+(of (app c+$^1$\verb+ d+$^2$\verb+) T+$^0$\verb+)+.
Concretely, \tedius{} replaces the bound names \verb+f+ and \verb+w+ with the level-annotated constants $c^1$ and $d^2$ performing the $\beta$-reductions
in a lazy way using an explicit substitution calculus
\`a la $\lambda\sigma$~\cite{DBLP:journals/jar/LiangNQ04}.
%The lazy propagation
%of explicit substitutions has typically poor memory performance.
In \rff{} such substitutions will be completely avoided.

% 
% \begin{table}
% \begin{center}
% \begin{tabular}{c@{~~}|@{~~}c}
% \begin{minipage}{5.0cm}
% \begin{verbatim}
% forall P [].
% forall P [X|XS] :-
%    P X, forall P XS.
% \end{verbatim}
% \end{minipage}
% &
% \begin{minipage}{5.0cm}
% \begin{verbatim}
% (pi Y\  Y :- pi x0\ r Y) =>
%    (pi x0\ q (x1\ X^0 x0 x1))
% \end{verbatim}
% \end{minipage}
% \end{tabular}
% \end{center}
% \caption{\label{example4} Example 4 (on the left): an higher order predicate whose definition is not in the pattern fragment. Example 5 (on the right): all clauses are in the \frag, but after two inferences the obtain term is not.}
% \end{table}
% 
% The pattern fragment, discovered by Dale Miller in~\cite{???}, is a
% well-behaved fragment of higher-order unification that is stable under \lp{}
% resolution. I.e. if the syntax of the program is in the fragment, all
% unification problems will only involve terms that are still in the fragment.
% Moreover, all unification problems in the pattern fragment admit a most
% general unifier, even in the flexible-flexible and flexible-rigid case. The
% fragment is easily defined as follows: a variable \verb+X+ can only be
% applied to a list of distinct names, and only if those names are not in the
% scope of \verb+X+.
% Example 1 is written in the pattern fragment: \verb+F+ is applied to \verb+x+,
% and \verb+F+ is universally quantified before \verb+x+, that therefore is not
% in the scope of \verb+F+. Example 2, instead, is not in the fragment because
% of \verb+F N+ where \verb+N+ is not a name. However, at the expense of some
% verbosity, it is possible to rewrite Example 2 to force it in the pattern
% fragment (see Table~\ref{example3}). The definition of higher order predicates
% like in Table~\ref{example4}, is, however, typically outside the pattern
% fragment.

\section{Bound variables}
\label{sec:dbl}

The last missing ingredient to define \rff{} and explain why it can be
implemented efficiently is to see how systems that manipulate $\lambda$-terms
accommodate
$\alpha$-equivalence. Bound variables are not represented by using
real names, but canonical ``names'' (hence $\alpha$-equivalence becomes syntactic equality).
De Bruijn introduced two, dual, naming schemas for $\lambda$-terms in~\cite{debruijnlevel}:
depth indexes (DBI) and levels
(DBL). In the former, that is the most widely adopted one, a variable is named $n$ if its binder is found by
crossing $n$ binders going in the direction of the root. In the latter a
variable named $n$ is bound by the $n$-th binder one encounters in the path
from the root to the variable.
In the following table we write the term $\lambda x.(\lambda y.\lambda z.f~x~y~z)~x$ and its reduct in the two notations:
%\begin{table}
\vspace{-0.5em}
\begin{center}
\begin{tabular}{r@{~~}c@{~$\to_\beta$~}c}
Indexes: & $\lambda x.(\lambda y.\lambda z.f~x_2~y_1~z_0)~x_0$ &
$\lambda x.\lambda z.f~x_1~x_1~z_0$ \\
Levels: & $\lambda x.(\lambda y.\lambda z.f~x_0~y_1~z_2)~x_0$ &
$\lambda x.\lambda z.f~x_0~x_0~z_1$ \\
\end{tabular}
\end{center}
\vspace{-0.5em}
%\end{table}
In both notations when a binder is removed and the corresponding variable
substituted some ``renaming'' (called lifting) is performed.
\tedius{} follows
a third approach that mixes the two, using indexes for variables bound in
the terms, and levels for variables bound in the context. The advantage
is that no lifting is required when moving a term under additional binders.
However, an expensive substitution of a level for an index is
required to push a binder to the context.
%\marginpar{this next paragraph can be dropped IMO}
%We believe the popularity of DBI comes from the
%fact that weak head normalization is easier to code: the argument of the
%redex, being at the top level of the term, is always closed and hence
%invariant by lifting.

In \elpi{} we crucially chose DBL because of the following three properties:
\begin{description}
\item[DBL1] $x_i$ in $\Gamma$ keeps the name $x_i$ in $\Gamma, \Delta$
\item[DBL2] the variables bound by $\Gamma$ in a $\beta$-redex keep their name in the reduct
\item[DBL3] when a binder is pushed to the context, the bound occurrences keep
 their name: $\Gamma \vdash $\verb+ pi \+$ x^n \ p(x^n) \leadsto \Gamma, x^n \vdash p(x^n)$
\end{description}
Another way to put it is that variables already pushed in the context
are treated \emph{exactly as constants}, and that the two notions of level
--- De Bruijn's and the position in the context introduced in Section~\ref{sec:ho} --- coincide.

%One important property of the pattern fragment is that $\beta$-reduction can
%only substitute names with other names. This property, however, is not
%easily exploitable and an implementation still needs to traverse the body
%of the abstraction to do the replacing. If we look at Example 1, we see that
%in order to type-check a $\lambda$-abstraction the operational semantics
%requires to first traverse all the body just to replace a bound name with a
%fresh one. This is a major source of inefficiency that is hardly justified.
%To cope with it, \tedius{} 2.0 invented the \emph{suspension calculus}~\cite{susp1,susp2}, a form of explicit substitution calculus that propagates the substitution lazily (and that works also outside the pattern fragment). Some benchmarks~\cite{susp3} concluded that adopting the suspension calculus provides a significant speedup over a naive implementation of susbtitution. Nevertheless, programs compiled with \tedius{} are quite slow and, at least in cases like Example~1, the whole idea of substituting bound names with fresh names, in the spirit of the locally nameless approach~\cite{???}, seems a waste of time.
%
%In~\ref{sec:fragment} we will identify a sub-fragment of the pattern fragment
%that we call~\frag. A clever implementation can completely avoid the traversal of the body of the abstraction during $\beta$-reduction, if the term to be reduced is in the~\frag.
%In the remaining sections we will present an interpreter
%for \lp, written in OCaml and called \elpi{} (Embedded \lp{} Interpreter), that exploits the \frag, and we will compare it to \tedius{} on a few benchmarks.
%We anticipate that \elpi{} is consistently faster than \tedius{} on every test, sometimes up to \CSC{XXXX} times.

\section{The reduction-free fragment \rff{}}\label{sec:fragment}
\lp{} is a truly higher order language: even clauses can be
passed around, unified, etc. Nevertheless this plays no role here, so
we exclude formulas from the syntax of terms.
Therefore, our terms are defined just by:
\vspace{-0.5em}
$$\begin{array}{l}
   t ::= x_i ~|~ X^j ~|~ \lambda x_i.t ~|~ t~t
\end{array}$$
\vspace{-0.1em}
Since variables follow the DBL representation, we do not have a case
for constants like \verb+app+ or \verb+lam+, that are represented as $x_i$ for some negative $i$.
Since the
level of a variable completely identifies it, when we write
$x_i \ldots x_{i+k}$ we mean $k$ distinct bound (i.e. $i \geq 0$) variables.
The superscript $j$ annotates unification variables with their
visibility range ($0 \leq j$, since all global constants are in range).
A variable $X^j$ has visibility of all names strictly
smaller than $j$. E.g. $X^1$ has visibility only of $\{\ldots,x_{-1},x_0\}$, and $X^3$ has
visibility of $\{\ldots,x_{-1},x_0,x_1,x_2\}$. Technically, when following
the De Bruijn convention, we could just write $\lambda x_i.t$ as $\lambda.t$.
We keep writing the name $x_i$ to ease reading.

\begin{mydef}[\rff{}]
A term is in the reduction-free fragment \rff{} iff
every occurrence of a unification variable $X^j$ is applied to
$x_j \ldots x_{j+k-1}$ for $k >= 0$.
\end{mydef}

We allow $k=0$ to accept variables that are not applied. A consequence of
the definition is that if a term is in \rff{} then all occurrences of
applications of unification variables
can be instantiated with a term closed
in an initial segment of the \lp{} context seen as a ordered list.
Examples: $X^2~x_2~x_3$ and $X^2$ are in the fragment; $X^2~x_3$ and
$X^2~x_3~x_2$ are not; $X^2~x_2~x_3$ can be instantiated with any term
closed in $\{\ldots,x_0,x_1,x_2,x_3\}$.

Observe that the programs in Example~\ref{example1} (when \verb+cbn+ is
rewritten to be in the pattern fragment as in Section~\ref{sec:ho}) are in
\rff{}. Also, every Prolog program is in \rff{}. As we will see
in~Section~\ref{sec:benchmarks}, a
type-checker for a dependently typed language
and evaluator based on a reduction machine are also naturally in \rff{},
showing that, in practice, the fragment is quite expressive.

% where $c,x,X$ range respectively over the set of constants, the set of universally quantified names (also called local constants) and the set of existentially quantified variables. Formulas $Q$ that occur in positive positions are called \emph{queries}. Formulas $P$ that occur in negative position are called \emph{clauses}. The logical atoms are captured by $T$. Note that, the language being
% higher order, there is no distinction between formulas and terms, and both queries and clauses can be passed as arguments to predicates. Note also that $\forall$ binds local constants in positive position, and existentially quantified variables in negative position, according to the logical equivalence
% $(\forall x.P) \Rightarrow Q \equiv \exists x.(P \Rightarrow Q)$. Disjunction
% is definable, but the derivational semantics is complete only if it is used in
% positive formulas (like for the existential quantifiers). We let $\mathcal{E}$ range over all syntactic categories above, and we call $\mathcal{E}$ an expression.
% 
% The concrete syntax differs from the abstract syntax in a few ways: \verb+pi x\+ and \verb+sigma x\+ are used in place of $\forall x.$ and $\exists x.$; $P \Rightarrow Q$ can be written as \verb+P => Q+ or as \verb+Q :- P+; conjunction is written using commas; $\lambda x.$ is written \verb+x\+. A program is just a list of clauses terminated by dots. All free uppercase names in a clause are implicitly universally quantified around the clause. The user writes a query to be resolved against a program. All free uppercase names in the query are also implicitly existentially quantified. The semantics of the pair program/clause is a new query obtained as the implication between the $n$-ary conjunction of all the clauses and the user provided query.
% 
% $\lambda x.A$ and $\forall x.Q$ bind $x$ respectively in $A$ and $Q$;
% $\forall X.P$ and $\exists X.Q$ bind $X$ respectively in $P$ and $Q$.
% Formulas are identified up to $\alpha$-equality as usual.
% 
% \paragraph{Definition and expressivity of the \frag}
% The De Bruijn level of a name $x$ in an expression is the number of binders to be crossed when traversing the expression from the root towards the leaves before finding the binder for $x$~\cite{debruijnlevel}. For example, in the following expression every variable $x_i$ has level $i$: $\forall x_0.\exists X^1.\lambda x_1.\lambda x_2. \exists X^3. p~(X^1~x^1) X^3$. Every expression can be
% $\alpha$-converted so that every name is renamed to $x_i$ where $i$ is its level. In the rest of the discussion we assume every term to be renamed in this way. Similarly, we can rename every variable $X$ to $X^j$ where $j$ is the smallest number such that $X^j$ has visibility of all names smaller than $j$. In the example above, $X^1$ has visibility only of $x^0$, and $X^3$ has visibility of $\{x^0,x^1,x^2\}$.
% 
% An expression is in the \frag{} iff every occurrence of an existentially quantified $X^j$ occurs applied to $x_j \ldots x_{j+k-1}$ for $k >= 0$.
% 
% Examples: $X^2~x_2~x_3$ and $X^2$ are in the fragment; $X^2~x_3$ and $X^2~x_3~x_2$ are not.

\begin{myprop}[Decidability of HO unification]
Being \rff{} included in the pattern-fragment \Ll{}, higher order unification is
decidable for \rff{}.
\end{myprop}
%In other words, each problem of the form $X^j~x_j\ldots x_{j+k-1} \equiv t$
%admits a MGU.

The most interesting property of \rff{}, which also justify its name, is:

\begin{myprop}[Constant time head $\beta$-reduction]
Let $\sigma$ be a \emph{valid} substitution for existentially quantified variables. Then the first $k-1$ head reductions of $(X^j~x_j \ldots x_{j+k-1}) \sigma$ can be computed in constant time.
\end{myprop}

A valid substitution assigns to $X^j$ a term $t$ of the right type (as in simply
typed $\lambda$-calculus) and such that the free variables of $t$ are
all visible by $X^j$ (all $x_i$ are such that $i < j$).
Therefore $X^j \sigma = \lambda x_j. \ldots \lambda x_{j+n}.t$ for some $n$.
Then
\begin{equation}\label{deffrag}(X^j~x_j \ldots x_{j+k-1}) \sigma
 = \left\{ \begin{array}{ll}
t~x_{j+n+1} \ldots x_{j+k-1} & \mbox{if $n+1 < k$} \\
\lambda x_{j+k}. \ldots \lambda x_{j+n}.t & \mbox{otherwise}
      \end{array} \right.\end{equation}
Thanks to property \textbf{DBL2}, Equation~\ref{deffrag} is
\emph{syntactical}: no lifting of $t$ is required.
Hence the $\beta$-reductions triggered by the substitution of $X^j$ take
constant time.

\begin{myprop}[Constant time unification]
A unification problem of the form $X^j~x_j\ldots x_{j+k-1} \equiv t$
can be frequently solved in constant time.
\end{myprop}

The unification problem $X^j~x_j\ldots x_{j+k-1} \equiv t$ can always be
rewritten as two simpler problems: $X^j \equiv \lambda x_j. \ldots \lambda x_{j+k-1}. Y^{j+k}$ and $Y^{j+k} \equiv t$ for a fresh $Y$.
The former is a trivial assignment that requires no check.
The latter can be implemented in constant time iff no occur-check is needed
for $X$ and if the level of the highest free variable in $t$ can be recovered
in $O(1)$ and is smaller than $j+k$. The recovery can be economically
implemented caching the maximum level in the term, that is something often
pre-computed on the input term in linear time.
Avoiding useless occur-check is a typical optimization of the Warren Abstract
Machine (WAM), e.g. when $X$ occurs linearly in the head of a clause.
%With such maximum level $l$ at hand the problem $Y^{j+k}
%\equiv t$ can be decided by simply comparing $j+k$ with $l$.
These properties enable us to implement the operational semantics of \verb+pi+
in constant time for terms in \rff{}.

We detail an example.
The first column gathers the fresh constants and extra clauses. The
second one shows the current goal(s) and the program clause that is
used to back chain.

\begin{center}
\small
\begin{tabular}{c|l}
Context & Goals and refreshed program clause \\\hline
& \verb+of (lam x+$_0$\verb+\lam x+$_1$\verb+\app x+$_0$\verb+ x+$_1$\verb+) T+$^0$ \\
& \verb+of (lam F+$^0$\verb+) (arr A+$^0$\verb+ B+$^0$\verb+) :- pi x+$_0$\verb+\ of x+$_0$\verb+ A+$^0$\verb+ => of (F+$^0$\verb+ x+$_0$\verb+) B+$^0$ \\\hline
\verb+ x+$_0$;\verb+(of x+$_0$\verb+ A+$^0$\verb+)+ & \verb+of (lam x+$_1$\verb+\app x+$_0$\verb+ x+$_1$\verb+) B+$^0$ \\
& \verb+of (lam G+$^1$\verb+) (arr C+$^1$\verb+ D+$^1$\verb+) :- pi x+$_1$\verb+\ of x+$_1$\verb+ C+$^1$\verb+ => of (G+$^1$\verb+ x+$_1$\verb+) D+$^1$\verb++ \\\hline
\verb+ x+$_0$;\verb+(of x+$_0$\verb+ A+$^0$\verb+)+ & \verb+of (app x+$_0$\verb+ x+$_1$\verb+) D+$^0$ \\
\verb+ x+$_1$;\verb+(of x+$_1$\verb+ C+$^0$\verb+)+ & \verb+of (app M+$^2$\verb+ N+$^2$\verb+) S+$^2$\verb+ :- of M+$^2$\verb+ (arr R+$^2$\verb+ S+$^2$\verb+), of N+$^2$\verb+ R+$^2$ \\\hline
\verb+ x+$_0$;\verb+(of x+$_0$\verb+ A+$^0$\verb+)+ & \verb+of x+$_0$\verb+ (arr R+$^2$\verb+ S+$^0$\verb+), of x+$_1$\verb+ R+$^2$ \\
\verb+ x+$_1$;\verb+(of x+$_1$\verb+ C+$^0$\verb+)+ & \verb+of x+$_0$\verb+ A+$^0$\hspace{4pt}\verb+        , of x+$_1$\verb+ C+$^0$ \\\hline
\end{tabular}
\end{center}

After the first step we obtain
\verb+F+$^0$\verb+:= x+$_0$\verb+\lam x+$_1$\verb+\app x+$_0$\verb+ x+$_1$;
\verb+T+$^0$\verb+:= arr A+$^0$\verb+ B+$^0$; the extra clause
about \verb+x+$_0$ in the context and a new subgoal.
Thanks to property \textbf{DBL3}, \verb+x+$_0$ has been pushed to the context
in constant time.
Note that the redex \verb+(F+$^0$\verb+ x+$_0$\verb+)+ is in \rff{} and
thanks to Equation~\ref{deffrag} head normalizes in constant time
to \verb+(lam x+$_1$\verb+\app x+$_0$\verb+ x+$_1$\verb+)+.
The same phenomenon arises in the second step,
where we obtain \verb+G+$^1$\verb+:= x+$_1$\verb+\app x+$_0$\verb+ x+$_1$
and we generate the redex \verb+(G+$^1$\verb+ x+$_1$\verb+)+.
Unification variables are refreshed in the context under with
the clause is used, e.g. \verb+C+ is placed at level 1 initially,
but in consequence to a unification step they may be \emph{pruned}
when occurring in a term assigned to a lower level unification
variable. Example: unifying
\verb+B+$^0$ with \verb+(arr C+$^1$\verb+ D+$^1$\verb+)+ prunes
\verb+C+ and \verb+D+ to level 0.

The choice of using DBL for bound variables is both an advantage and a
complication here.
Clauses containing no bound variables, like
\verb+(of x+$_0$\verb+ A+$^0$\verb+)+, require no processing thanks to
\textbf{DBL1}: they can be indexed as they are, since the name \verb+x+$_0$
is stable.
The drawback is that clauses with bound variables, like the one 
used in the first two back chains, need to be lifted: the first time the
bound variable is named \verb+x+$_0$,
while the second time \verb+x+$_1$.
Luckily, this renaming, because of property \textbf{DBL1},
can be performed in constant time using the very same machinery one uses to
refresh the unification variables.
E.g. when the WAM unifies the head of a clauses it assigns 
fresh stack cells: the clause is not really refreshed and the stack
pointer is simply incremented. One can represent the locally bound variable as
an extra unification variable, and initialize, when \verb+pi+ is crossed, the
corresponding stack cell to the first \verb+x+$_i$ free in the context.

\paragraph{Stability of \rff{}.}
Unlike \Ll{}, \rff{} is not stable under \lp{} resolution: a clause that
contains only terms in \rff{} may generate terms outside the fragment.
Therefore an implementation must handle both terms in \rff{}, with their
efficient computation rules, and terms outside the fragment. Our limited
experience so far, however, is that several programs initially written in the
fragment remains in the fragment during computation, or they can be slightly
modified to achieve that property.


% The term $X^j~x_j\ldots x_{j+k-1}$ can always be rewritten as
% $Y^{j+k}$ instantiating $X^j$ with $\lambda x_j. \ldots \lambda x_{j+k-1}. X^{j+k}$ for a fresh existentially quantified variable $X^{j+k}$. Therefore every
% unification problem in the \frag{} can be reduced in linear time in $k$ to a
% problem of the kind $X^{j+k} \equiv \mathcal{E}$, that is immediately solved
% by instantiating $X^{j+k}$ with $\mathcal{E}$ if the free names of $E$ are
% known to be a subset of $\{x_0,\ldots,x_{j+k-1}\}$.
% 
% Note that, once all $\beta$-redex that occur syntactically in the term are
% fired at compile time, or if explicitly written $\beta$-redexes are forbidden,
% then every $\beta$-redex in the computation is triggered by the substitution
% of an existentially quantified variable. Therefore, if the program is and
% remains in \rff{}, then all $\beta$-reductions can be implemented in
% constant time.
% 
% Finally, another benefit of working with De Bruijn levels is that the
% $\forall$-introduction rule can also be implemented in constant time as well
% because there is no need to replace the bound variable with a fresh name.
% The price to pay is an additional complexity in the backchain rule: when
% a clause is selected, the clause needs to be $\alpha$-converted --- lifted
% in De Bruijn indexes/levels terminology --- to move it in the current scope.
% For example, if the query is \verb+ pi x_0\ r x_0+ and \verb+r X :- pi x_0\ pi x_1\ p x_0 x_1 X+ is a clause, after two inferences the new query becomes
% \verb+pi x_1\ pi x_2\ p x_1 x_2 x_0+ because the introduction rule for
% \verb+pi+ has already fixed the name \verb+x_0+.
% 
% \paragraph{Instability of the \frag}
% Unlike the pattern fragment, the \frag{} is unstable for computation.
% I.e. there are simple examples where the program and the initial query are
% written in the fragment, but during execution we generate terms outside the
% fragment (but still in the pattern fragment). This is a consequence of the
% need for lifting clauses during backchaining.
% Consider example~5 in Table~\ref{example4}.
% After two inference steps the term \verb+(x1\ X^0 x0 x1)+, that is in
% the fragment and assigned to \verb+Y+, is moved under \verb+pi x0+ becoming
% \verb+r (x2\ X^0 x0 x2)+. The latter is no longer in the fragment.
% 
% Therefore, and contrarily to what happens with the pattern fragment, an
% implementation must handle both terms in the fragment, with their efficient
% computation rules, and terms outside the fragment. Our limited experience so
% far, however, is that several programs initially written in the fragment
% remains in the fragment during computation, or they can be slightly modified
% to achieve that property.
% 
% \section{\elpi: an Embedded \lp{} Interpreter.}\label{sec:elpi}
% The current version of \elpi, together with the benchmarks presented in
% the next section, can be downloaded at \CSC{\url{http://xxxxx}}. The intepreter
% is entirely written in OCaml and it is \CSC{XXX} lines long. It augments the language presented in the paper with Prolog-style cuts and a few custom predicates for printing. The name is due to the fact that we eventually plan to augment the language with constraints and then embed it in the implementation of interactive theorem provers like Coq, that are often written in OCaml.
% 
% We carefully wrote the code so that unrechable terms in \lp{} are encoded by unreachable terms in OCaml. Therefore we inherit garbage collection from the OCaml runtime. The algebraic datatype used to encode terms has different constructors for occurrences of terms in the \frag, and for occurrences outside the fragment. In particular an occurrence $X^j x_j \ldots x_{j+k-1}$ is simply represented as a triple $\langle r,j,k\rangle$ where $r$ is a reference to the term
% $X$ is instantiated to, or to a dummy term if $X$ is still unbound. Finally, we carefully cherry picked some design decisions from the WAM, implementing our own variations in several places. The variations have been made necessary to decrease the pressure on the garbage collector of OCaml, that is responsible for a significant percentage of the running time. For example, the heap and the stack are not organized as large arrays of mostly unused cells because otherwise the garbage collector needs to traverse the arrays at every minor and major collection. Similarly, we employ a mix of mutable and persistent data structures, avoiding mutable structures for data that is likely to survive, in order to avoid the penalty of the write barrier on data in the old generation~\cite{???}.
% Further details will be reported in a future work due to lack of space.

\section{Assessment and conclusions}\label{sec:benchmarks}

We assess %the performances of 
\elpi{} on a set of synthetic benchmarks and
a real application. Synthetic benchmarks are divided into three groups:
first order programs from the Aquarius test suite (%the
crypto-multiplication, $\mu$-puzzle,
generalized eight queens problem and the Einstein's zebra puzzle);
higher order programs falling in \rff{}; and an higher
order program falling outside \rff{} taken from the test suite of
\tedius{} normalizing expressions in the SKI calculus.

The programs in \rff{} are respectively type checking lambda terms using
the \verb+of+ program of Example~\ref{example1} and reducing expressions like
$5^5$ in the syntax of Church numerals using a call by value/name (CBV/CBN)
strategy.  The typeof test was specifically conceived to measure the cost of
moving under binders: the type checked terms, projections, are mainly made of
binders.

\begin{center}
  \scriptsize 
  \begin{tabular}{|p{1.5cm}||c|r||c|r||c|c|}
    \hline
      \multicolumn{1}{|c||}{Test} &
      \multicolumn{2}{|c||}{ELPI} &
      \multicolumn{2}{|c||}{\tedius{}} &
      \multicolumn{2}{|c|}{ELPI/\tedius{}} \\
    \hline
    &  time (s)     & space (Kb)  & time (s) & space (Kb) &  time & space \\
    \hline
    \hline
    crypto-mult &  3.48 & 27,632  & 6.59 & 18,048 &  0.52 & 1.53 \\
    \hline    
    $\mu$-puzzle &  1.82 & 5,684 &  3.62 & 50,076 &  0.50 & 0.11 \\
    \hline
    queens &  1.41  & 108,324 &  2.02 & 69,968 &  0.69 & 1.54 \\
    \hline    
    zebra &  0.85 & 7,008 &  1.89 & 8,412 &  0.44 & 0.83 \\
    \hline     
    \hline
    typeof &  0.27 & 8,872 &  5.64 & 239,892 &  0.04 & 0.03 \\
    \hline
    reduce\_cbv &  0.15 & 7,248 &  11.11 & 57,404  & 0.01 & 0.12 \\
    \hline
    reduce\_cbn &  0.33 & 8,968 &  0.81 & 102,896  & 0.40 & 0.08 \\
    \hline
    %reduce\_cbv\_nocopy & ... & ... &  11.25 & 50,892  & ... & ... \\
    %\hline
    %reduce\_cbn\_nocopy & ... & ... &  0.52 & 70,760  & ... & ... \\
    %\hline
    \hline
    SKI &  1.32 & 15,472 &  2.68 & 8,896  & 0.49 & 2.73 \\
    \hline
    
  \end{tabular}
\end{center}

The table shows that \elpi{} shines on programs in \rff{}, and compares well
outside it. The alternating performance \tedius{} on the reduction tests has
to be attributed to the explicit substitutions machinery~\cite{DBLP:journals/jar/LiangNQ04} \`a la $\lambda\sigma$ when employed to cross binders:
by its very nature $\lambda\sigma$ fits well a lazy reduction strategy like
CBN (even if some space
overhead is visible).  On the contrary $\lambda\sigma$ is counterproductive
in the CBV case since the program, by fully traversing the redex argument,
systematically pushes the suspended substitution to the leaves of the
term, completely defeating the purpose of the entire machinery (i.e. if the
substitution has to be performed, there is not gain is delaying it).
If one makes \tedius{} artificially
push explicit substitutions in the CBN case too, he halves memory consumption
but degrades the performances by 10 seconds, confirming the
time we see in the CBV case is dominated by the overhead of explicit substitutions. By avoiding substitution when crossing binders
\elpi{} is not only faster, but also more predictable performance wise: as one
expects CBV is faster than CBN in computing the normal form of $5^5$ since
it avoids duplicating non normal terms.
% \tedius{} is quick on the reduce\_cbn test only because CBN does not require to traverse the arguments and thus the explicit substitutions are
% not propagated. Inserting in the test a seamlessy innocuous printing of the
% argument triggers propagation and makes the CBN case as slow as the CBV case.

% \marginpar{maybe shrink this speculation}
% It is hard to understand why \elpi{} is faster than \tedius{}
% outside \rff{}, since all the optimizations we implemented are inspired by the
% literature about the WAM, on which \tedius{} is also based.  Our guess is that
% the OCaml garbage collector is responsible for
% the difference. Especially because we heavily optimized our code to lower the
% pressure on it in order to let it perform at the best of its capabilities.

The real application we present is a checker for the formal system
$\lambda\delta$~\cite{lambdadeltaJ1,lambdadeltaJ3a} Such
checker is able to validate the proof terms of the formalization of Landau's
``Grundlagen''~\cite{Jut79} done in Automath. The reference checker for
$\lambda\delta$, named Helena, has been implemented in OCaml.
Our \lp{} implementation follows it closely, and
naturally falls in \rff{}.
Nevertheless, the \lp{} code is much simpler that the corresponding OCaml code
and consists of just 52 clauses. In particular, the automatic
context management provided by the \lp{} runtime is used in place of the
environment that the OCaml code has to handle explicitly.

The ``Grundlagen'' is a theory comprising definitions and proofs
 %32 declarations and 6879
%definitions, for a total of 
for a total of 6911 items (circa 8MB of data).
%This amount of data seems to overwhelm the capabilities of
\tedius{} seems to have, by design, a limit in the size of the heap
(circa 256MB) that limits it to the verification of the first 2615 items.
\marginpar{The DATA says 1GB!!!}
In the table we compare pre-processing (Pre) time like parsing, compilation
or elaboration, and verification (Ver).
We compare \elpi{} with Helena, \tedius{}, and Coq.
The Coq system implements a
type checker for a $\lambda$-calculus strictly more expressive then the one
required in order to check the ``Grundlagen'', hence can check the proof terms
directly but surely incurs in some overhead. We use its timings mainly as a
reference for the order of magnitude of a state-of-the-art ITP.

\begin{center}
  \scriptsize 
\begin{tabular}{|l|c|c|c|}
\hline
\multicolumn{4}{|c|}{Time (s) for 2615 items only}\\
\hline
 & \elpi          & Teyjus  & \elpi/Teyjus        \\
\hline
Pre      & 2.55 & 49.57 & 0.05 \\
\hline
Ver      & 3.06 & 203.36 & 0.02 \\ \hline
RAM (Mb) & 91,628 & 1,072,092 & 0.09 \\
\hline
\end{tabular}
~~
\begin{tabular}{|l|c|c|c|c|c|}
\hline
\multicolumn{6}{|c|}{Time (s) for all 6911 items}\\
\hline
Task                   &\multicolumn{2}{|c|}{Helena}         & \elpi          & \multicolumn{2}{|c|}{Coq}            \\
                 &interp. & comp.& &interp. & comp.\\ \hline
Pre & 2.42 & 0.41 & 9.04 & 49.28 & 8.83 \\
\hline
Ver & 4.40 & 0.33 & 13.90 & 7.21 & 1.19\\ % coqchk.opt: 3.90
\hline
\end{tabular}
\end{center}

The pre-processing phase for \tedius{} required us to split the data in 28
modules, since we observed a non-linear time growth when compiling a single
module. 
% Splitting the compilation in even smaller modules may lower the
% pre-processing time, but, 
To our understanding this should not have affected the execution time.

Our conclusion is that \rff{} admits a very efficient implementation.
\elpi{} is work in progress, available at \url{http://lpcic.gforge.inria.fr} together with all the benchmarks presented in the paper.

\marginpar{in the biblio we can get rid of odd order thm if needed}
\bibliographystyle{plain}
\bibliography{reference}

\end{document} 
