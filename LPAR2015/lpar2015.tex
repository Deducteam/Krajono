% LLNCStmpl.tex
% Template file to use for LLNCS papers prepared in LaTeX
%websites for more information: http://www.springer.com
%http://www.springer.com/lncs

\documentclass{llncs}
%Use this line instead if you want to use running heads (i.e. headers on each page):
%\documentclass[runningheads]{llncs}
\usepackage{amssymb}
\usepackage{amsmath}
\usepackage{graphicx}
\usepackage{url}
\usepackage{color}
\usepackage{float}
\usepackage{tabularx}
\usepackage{multirow}
\setlength{\extrarowheight}{1pt}
\usepackage{fancyvrb}

\floatstyle{plain} \newfloat{example}{thp}{lop} \floatname{example}{\textbf{Example}}
\newtheorem{mydef}{Definition}
\newtheorem{myprop}{Property}
\begin{document}
\title{ELPI: fast, Embeddable, \lp{} Interpreter}

%If you're using runningheads you can add an abreviated title for the running head on odd pages using the following
%\titlerunning{abreviated title goes here}
%and an alternative title for the table of contents:
%\toctitle{table of contents title}

%\subtitle{Subtitle Goes Here}

%For a single author
%\author{Author Name}

%For multiple authors:

\author{Cvetan~Dunchev,$^1$~Ferruccio~Guidi,$^1$~Claudio~Sacerdoti~Coen,$^1$~Enrico~Tassi$^2$}


%If using runnningheads you can abbreviate the author name on even pages:
%\authorrunning{abbreviated author name}
%and you can change the author name in the table of contents
%\tocauthor{enhanced author name}

%For a single institute
\institute{Department of Computer Science,
University of Bologna,~%\\ Mura Anteo Zamboni 7, 40127 Bologna, Italy \\
\email{name.surname@unibo.it}
\and Inria Sophia-Antipolis,~%\\ 2004 route des Lucioles - BP 93, 06902 Sophia Antipolis Cedex, France
\email{name.surname@inria.fr}}

% If authors are from different institutes
%\institute{First Institute Name \email{email address} \and Second Institute Name\thanks{Thank you to...} \email{email address}}

%to remove your email just remove '\email{email address}'
% you can also remove the thanks footnote by removing '\thanks{Thank you to...}'

\newcommand{\frag}{Reduction-Free Fragment}
\newcommand{\lp}{$\lambda$Prolog}
\newcommand{\Ll}{\ensuremath{\mathcal{L}_\lambda}}
\newcommand{\elpi}{ELPI}
\newcommand{\tedius}{Teyjus}
\newcommand{\CSC}[1]{\textcolor{red}{#1}}
\newcommand{\FG}[1]{\textcolor{magenta}{#1}}

\maketitle

\begin{abstract}
We present a new interpreter for \lp{} that runs consistently faster than
the byte code compiled by \tedius{}, that is believed to be the best
available implementation for \lp. 
The key insight is the identification of a fragment of
the language, which we call \frag{}, that occurs quite naturally in \lp{}
programs and that admits constant time reduction and unification rules.
% The implementation exploits De Bruijn levels and no explicit substitutions,
% whereas \tedius{} is based on De Bruijn indexes and explicit substitutions
% (the suspension calculus).
\end{abstract}

\section{Introduction}% and State of the Art}
\lp{} is a logic programming language based on an intuitionistic fragment of
Church's Simple Theory of Types. An extensive introduction to the language
with examples can be found in~\cite{Miller:2012:PHL:2331097}. \tedius{}
\cite{DBLP:conf/cade/NadathurM99,DBLP:journals/corr/abs-0911-5203} is a
compiler for \lp{} %implemented by Gopalan Nadathur and others
that is considered to be the fastest
implementation of the language. 
%Previous slower implementations are described
%in~\cite{}. \CSC{OTHER IMPLEMENTATIONS IN ISABELLE ETC. MISSING}
The main difference with respect to Prolog is that \lp{} manipulates
$\lambda$-tree expressions, i.e. syntax containing binders. Therefore the
natural application of \lp{} is meta-programming (see~\cite{LPAZ} for
an interesting discussion), including: automatic generation of programs from
specifications; animation of operational semantics;
program transformations and implementation of type checking algorithms.

Via the Curry-Howard isomorphism, a type-checker is a proof-checker, the main
component of an interactive theorem prover (ITP). Indeed the motivation of our
interest in \lp{} is that we are looking for the best 
language to implement the so called \emph{elaborator} component of an ITP.
The elaborator is used to type check the terms input by
the user.  Such data, for conciseness reasons, is typically incomplete and
the ITP is expected to infer what is missing.  The possibility to
extend the built in elaborator with user provided ``logic programs'' (in the
form of type classes or unification hints) to infer the missing pieces of
information turned out to be a key ingredient in successful formalizations
like~\cite{gonthier:hal-00816699}.  Embedding a \lp{} interpreter in an ITP 
would enable the elaborator and its extensions to be expressed in the same,
high level, language.  A crucial requisite for this plan to be realistic is
the efficiency of the \lp{} interpreter.

In this paper we introduce ELPI, a fast \lp{} interpreter that being written
entirely in OCaml can be easily embedded in OCaml softwares, like Coq.
In particular we focus on the insight that makes ELPI fast when dealing with
binders by identifying a reduction free fragment (RFF) of \lp{} that, if
implemented correctly, admits constant-time unification and reduction
operations.
We analyze the role  of $\beta$-reduction in Section~\ref{sec:beta} and
higher order unification in Section~\ref{sec:ho}; we discuss bound names
representations in Section~\ref{sec:dbl}; we define the RFF 
in Section~\ref{sec:fragment} and we asses the results in
Section~\ref{sec:benchmarks}.

% We don't mention constraints here, too much vapor ware IMO.
%\lp{} with Constraints (\`a la CLP) is the best choice.

\section{The two roles of $\beta$-reduction in \lp{}}
\label{sec:beta}

\begin{example}[b]
\begin{center}
\begin{tabular}{cc}
\begin{minipage}{4.8cm}
\begin{Verbatim}[numbers=left,numbersep=1pt,frame=leftline]
of (app M N) B :-
  of M (arr A B), of N A.
of (lam F) (arr A B) :-
  pi x\ of x A => of (F x) B.
\end{Verbatim}
\end{minipage}~~
&
~~~\begin{minipage}{6.5cm}
\begin{Verbatim}[numbers=left,firstnumber=5,numbersep=1pt,frame=leftline]
cbn (lam F) (lam F).
cbn (app (lam F) N) M :- cbn (F N) M.
cbn (app M N) R :-
  cbn M (lam F), cbn (app (lam F) N) R.
\end{Verbatim}
\end{minipage}
\end{tabular}
\end{center}
\caption{\label{example1} Type checker and Weak CBN for simply typed $\lambda$-calculus.}
\end{example}

We introduce \lp{} and discuss the role of $\beta$-reduction on the
Example~\ref{example1}. The $\lambda$-term %$\Delta$ defined as 
$(\lambda x.xx)$ is encoded %in $\lambda$-tree syntax 
as \verb+(lam (x\ app x x))+ where \verb+x\F+ is
the $\lambda$-abstraction of \lp{}, that binds \verb+x+ in \verb+F+, and
\verb+lam+ is the constructor for object-level abstraction, that builds
a term of type $\mathcal{T}$ from a function of type
$\mathcal{T} \to \mathcal{T}$, with $\mathcal{T}$ the type of representations
of $\lambda$-terms. \verb+app+ takes two terms of type $\mathcal{T}$ and builds
their object-level application of type $\mathcal{T}$.  Following the tradition of Prolog, capitals letters denote unification variables.

The second clause for the \verb+of+ predicate shows a recurrent pattern in
\lp: in order to analyze an higher order term, one needs to recurse
under a binder. This is achieved
exploiting the forall quantifier \verb+pi x\G+ together with logical
implication \verb+F => G+. Operationally: the forall quantifier declares a
new local constant \verb+x+, meant to be fresh in the entire program;
logical implication temporarily
augments the program with the new formula \verb+F+ about \verb+x+.
Denotationally, these are just the standard rules for introduction of
implication and the universal quantifier.

Note that the functional (sub-)term
\verb+F+ is applied to the fresh constant \verb+x+.  Being \verb+F+ a
function, the $\beta$-redex \verb+(F x)+, once reduced, denotes the body of
our object-level function where the bound variable is replaced by the fresh
constant \verb+x+.
The implication is used to
assume \verb+A+ to be the type of \verb+x+, in order to prove that the body of
the abstraction has type \verb+B+ and therefore the whole abstraction has type
\verb+(arr A B)+ (i.e. $A \to B$). Note that, unlike in
the standard presentation of the typing rules, we do not need to manipulate an
explicit context $\Gamma$ to type the free variables. Instead the assumptions
of the form \verb+(of x A)+ are just added to the program's clauses, and \lp{}
takes care of dropping them when \verb+x+ goes out of scope.
Example: if the initial goal is
\verb+(of (lam (w\ app w w)) T)+ by applying the second clause we assign
\verb+(arr A B)+ to \verb+T+ and generate
a new goal \verb+(of (app c c) B)+ (where \verb+c+ is the fresh constant
substituted for \verb+w+) to be solved with the extra clause \verb+(of c A)+
at disposal.

In this first example, the meta-level $\beta$-reduction is only employed
to inspect a term under a binder by replacing the bound name with a fresh
constant.  The second example
shows a radically different pattern: in order to implement object-level
substitution --- and thus object-level $\beta$-reduction --- we use the
meta-level $\beta$-reduction. E.g. if \verb+F+ is \verb+(w\ app w w)+
then \verb+(F N)+ reduces to \verb+(app N N)+.  Note that in this case
$\beta$-reduction is fully general, because it replaces a name with a term.
This distinction is crucial in the definition of the RFF in
Section~\ref{sec:fragment}.

\section{Higher Order unification}% and captures}
\label{sec:ho}

Unification in the presence of binders raises two %theoretical and practical
problems. First, the absence of most general unifiers (MGUs) makes
one of the primitive operations of \lp{} very delicate.  Second, one has
to find a way to avoid captures, i.e. check that unification variables
are instantiated with terms containing only bound variable in their scope.

To cope with the absence of MGUs, Dale Miller identified
in~\cite{Miller91alogic} a well-behaved fragment (\Ll{}) of higher-order (HO)
unification that admits MGUs and is stable under \lp{} resolution. 
The restriction defining \Ll{} is that unification variables can
only be applied to (distinct) variables (i.e. not arbitrary terms) that are
not already in the scope of the variable.
Such fragment can effectively serve as a primitive for a
programming language and indeed \tedius{} 2.0 is built around this fragment:
no attempt to enumerate all possible unifiers is performed, and unification
%\marginpar{cite huet? no space IMO}
problems falling outside \Ll{} are just delayed.  Many interesting \lp{}
programs can be rewritten to fall in the fragment. For example, we can
make \verb+cbn+ of Example~\ref{example1} stay in \Ll{} by replacing
line 6 (that contains the offending \verb+(F N)+ term) with the following code:
\vspace{-0.4em}
\begin{center}
\small
\begin{minipage}{10cm}
\begin{Verbatim}[numbers=left,numbersep=1pt,frame=leftline]
cbn (app (lam F) N) M :- subst F N B, cbn B M.
subst F N B :- pi x\ copy x N => copy (F x) B.
copy (lam F1) (lam F2) :- pi x\ copy x x => copy (F1 x) (F2 x).
copy (app M1 N1) (app M2 N2) :- copy M1 M2, copy N1 N2.
\end{Verbatim}
\end{minipage}
\end{center}
\vspace{-0.3em}
The idea of \verb+subst+ is that the term \verb+F+ is recursively copied in
the following way: each bound variable is copied in itself but for the top one
that is replaced by \verb+N+.
The interested reader can find an longer discussion about \verb+copy+
in~\cite[page 199]{Miller:2012:PHL:2331097}.  
The \verb+of+ program falls naturally in \Ll{}, since \verb+F+ is only applied
the fresh variable \verb+x+ (all unification variables in a \lp{} program are
implicitly existentially bound in front of the clause, so \verb+F+ does not
see \verb+x+).  The same holds for \verb+copy+.

To correctly implement HO unification, even in the restricted \Ll{} fragment,
one typically tracks the \emph{level} of unification variables an fresh constants.  The proof theoretic interpretation of a \lp{} execution as an intuitionistic proof gives the following reading: unification is taking place under
a mixed prefix of $\forall$ and $\exists$ quantifiers; their order
determines if a unification variable (an existential) can be assigned to
(proved by) a term that contain a universally quantified variable.
E.g. $\forall x,\exists Y, Y = x$ is always provable while
$\exists Y,\forall x, Y = x$ is not.
\marginpar{rephrase with symbols}
Whenever a clause is used, its unification variables are declared at
a level that corresponding the length of the current ``context'', and whenever
a fresh constant is created (for the \verb+pi+ quantifier) the context
is extended and the constant is placed at such level.  From now on we will
write levels in superscript.  If we run the program
\marginpar{sketched}
\verb+(of (lam f\lam w\app f w) T)+ after two steps the goal is
\verb+(of (app c+$^1$\verb+ d+$^2$\verb+) T+$^0$\verb+)+.
Replacing \verb+f+ and \verb+w+ by fresh constant annotated with a level is the
implementation technique adopted by \tedius{} to make unification able to
check levels correctly.  As an optimization \tedius{} performs the beta
reduction in a lazy way using an explicit substitution calculus.  In the RFF we
will be able to completely avoid such substitution.


% 
% \begin{table}
% \begin{center}
% \begin{tabular}{c@{~~}|@{~~}c}
% \begin{minipage}{5.0cm}
% \begin{verbatim}
% forall P [].
% forall P [X|XS] :-
%    P X, forall P XS.
% \end{verbatim}
% \end{minipage}
% &
% \begin{minipage}{5.0cm}
% \begin{verbatim}
% (pi Y\  Y :- pi x0\ r Y) =>
%    (pi x0\ q (x1\ X^0 x0 x1))
% \end{verbatim}
% \end{minipage}
% \end{tabular}
% \end{center}
% \caption{\label{example4} Example 4 (on the left): an higher order predicate whose definition is not in the pattern fragment. Example 5 (on the right): all clauses are in the \frag, but after two inferences the obtain term is not.}
% \end{table}
% 
% The pattern fragment, discovered by Dale Miller in~\cite{???}, is a
% well-behaved fragment of higher-order unification that is stable under \lp{}
% resolution. I.e. if the syntax of the program is in the fragment, all
% unification problems will only involve terms that are still in the fragment.
% Moreover, all unification problems in the pattern fragment admit a most
% general unifier, even in the flexible-flexible and flexible-rigid case. The
% fragment is easily defined as follows: a variable \verb+X+ can only be
% applied to a list of distinct names, and only if those names are not in the
% scope of \verb+X+.
% Example 1 is written in the pattern fragment: \verb+F+ is applied to \verb+x+,
% and \verb+F+ is universally quantified before \verb+x+, that therefore is not
% in the scope of \verb+F+. Example 2, instead, is not in the fragment because
% of \verb+F N+ where \verb+N+ is not a name. However, at the expense of some
% verbosity, it is possible to rewrite Example 2 to force it in the pattern
% fragment (see Table~\ref{example3}). The definition of higher order predicates
% like in Table~\ref{example4}, is, however, typically outside the pattern
% fragment.

\section{Bound variables}
\label{sec:dbl}

The last missing ingredient to define the RFF and explain why it can be
implemented efficiently is to see how systems manipulating $\lambda$-terms
accommodate
$\alpha$-equivalence.  Bound variables are not represented by using
real names, but canonical ``names'' (or better numbers).  De Bruijn introduced
two, dual, naming schemas for lambda terms in~\cite{debruijnlevel}:
indexes (DBI) and levels
(DBL).  In the former a variable is named $n$ if its binder is found by
crossing $n$ binders going in the direction of the root.  In the latter a
variable named $n$ is bound by the $n$-th binder one encounters in the path
from the root to the variable.
In the following table we write the term $\lambda x.(\lambda y.\lambda z.f~x~y~z)~x$ and its reduct in the two notations:
%\begin{table}
\vspace{-0.8em}
\begin{center}
\begin{tabular}{r@{~~}c@{~$\to_\beta$~}c}
Indexes: & $\lambda x.(\lambda y.\lambda z.f~x_2~y_1~z_0)~x_0$ &
$\lambda x.\lambda z.f~x_1~x_1~z_0$  \\
Levels: & $\lambda x.(\lambda y.\lambda z.f~x_0~y_1~z_2)~x_0$ &
$\lambda x.\lambda z.f~x_0~x_0~z_1$ \\
\end{tabular}
\end{center}
\vspace{-0.2em}
%\end{table}
In both notations when a binder is removed and the corresponding variable
substituted some ``renaming'' (called lifting) has to be performed.  
The DBI convention is way more popular than the DBL one and \tedius{}
indeed adopts that schema.  We believe the popularity of DBI comes from the
fact that weak head normalization is easier to code: the argument of the
redex, being at the top level of the term, is always closed and hence
invariant by lifting.

In \elpi{} we chose DBL because of the following two properties:
\begin{description}
\item[DBL1] the name of an occurrence of a variable does not depend on the (extra) context under which it occurs, i.e. $x$ is always named $x_0$;
\item[DBL2]
when $\beta$-reduction occurs under a context the variables bound
in such context do not change name. In our example, since the reduction occurs
under the binder for $x$, $x$ is named $x_0$ in the initial and in the reduct
term.
\end{description}
Another way to put it is that variables already pushed in the context
are treated \emph{exactly as constants}, and their name is the level at
which the occur in the context.  From now on we subscript variables
with their name in DBL convention.

%One important property of the pattern fragment is that $\beta$-reduction can
%only substitute names with other names. This property, however, is not
%easily exploitable and an implementation still needs to traverse the body
%of the abstraction to do the replacing. If we look at Example 1, we see that
%in order to type-check a $\lambda$-abstraction the operational semantics
%requires to first traverse all the body just to replace a bound name with a
%fresh one. This is a major source of inefficiency that is hardly justified.
%To cope with it, \tedius{} 2.0 invented the \emph{suspension calculus}~\cite{susp1,susp2}, a form of explicit substitution calculus that propagates the substitution lazily (and that works also outside the pattern fragment). Some benchmarks~\cite{susp3} concluded that adopting the suspension calculus provides a significant speedup over a naive implementation of susbtitution. Nevertheless, programs compiled with \tedius{} are quite slow and, at least in cases like Example~1, the whole idea of substituting bound names with fresh names, in the spirit of the locally nameless approach~\cite{???}, seems a waste of time.
%
%In~\ref{sec:fragment} we will identify a sub-fragment of the pattern fragment
%that we call~\frag. A clever implementation can completely avoid the traversal of the body of the abstraction during $\beta$-reduction, if the term to be reduced is in the~\frag.
%In the remaining sections we will present an interpreter
%for \lp, written in OCaml and called \elpi{} (Embedded \lp{} Interpreter), that exploits the \frag, and we will compare it to \tedius{} on a few benchmarks.
%We anticipate that \elpi{} is consistently faster than \tedius{} on every test, sometimes up to \CSC{XXXX} times.

\section{The~\frag.}\label{sec:fragment}
\lp{} is a truly higher order language: even clauses can be
passed around, unified, etc.  Nevertheless this plays no role here, so
we exclude from the syntax of terms the one of formulas.
\vspace{-0.5em}
$$\begin{array}{l}
   t ::= x_i ~|~ X^j ~|~ \lambda x_i.t ~|~ t~t
\end{array}$$
\vspace{-0.1em}
Since variables follow the DBL representation, we don't have a case
for constants: when $i < 0$ then $x_i$ represents a global constant,
like \verb+app+ or \verb+lam+ in Example~\ref{example1}.  Since the
level of a variable completely identifies it, when we write
$x_i \ldots x_{i+k}$ we mean $k$ distinct bound (i.e. $i \geq 0$) variables.
The superscript $j$ annotates unification variables with their
visibility range ($0 \leq j$, since all global constants are in range).  
A variable $X^j$ has visibility of all names strictly
smaller than $j$. E.g.  $X^1$ has visibility only of $x^0$, and $X^3$ has
visibility of $\{x^0,x^1,x^2\}$.  Technically a binder needs no
name when bound variables are named following a De Bruijn convention.
Still we write it to ease reading.

\begin{mydef}[RFF]
A term is in the Reduction Free Fragment iff
every occurrence of a unification variable $X^j$ is applied to
$x_j \ldots x_{j+k-1}$ for $k >= 0$.
\end{mydef}

Note that when $k$ is $0$ the variable is not applied.  Another way to
put it is that a term is in the RFF iff all unification variables see
a (complete, no hole) prefix of the \lp{} context seen as a ordered list.
Examples: $X^2~x_2~x_3$ and $X^2$ are in the fragment; $X^2~x_3$ and
$X^2~x_3~x_2$ are not.

Observe that the programs in Example~\ref{example1} (when \verb+cbn+ is
rewritten to be in the pattern fragment as in Section~\ref{sec:ho}) are in
the RFF.  Also, every Prolog program is in the RFF. As we will see
in~Section~\ref{sec:benchmarks}, a
type-checker for a dependently typed language
and evaluator based on a reduction machine are also naturally in RFF,
showing that, in practice, the fragment is quite expressive.


% where $c,x,X$ range respectively over the set of constants, the set of universally quantified names (also called local constants) and the set of existentially quantified variables. Formulas $Q$ that occur in positive positions are called \emph{queries}. Formulas $P$ that occur in negative position are called \emph{clauses}. The logical atoms are captured by $T$. Note that, the language being
% higher order, there is no distinction between formulas and terms, and both queries and clauses can be passed as arguments to predicates. Note also that $\forall$ binds local constants in positive position, and existentially quantified variables in negative position, according to the logical equivalence
% $(\forall x.P) \Rightarrow Q \equiv \exists x.(P \Rightarrow Q)$. Disjunction
% is definable, but the derivational semantics is complete only if it is used in
% positive formulas (like for the existential quantifiers). We let $\mathcal{E}$ range over all syntactic categories above, and we call $\mathcal{E}$ an expression.
% 
% The concrete syntax differs from the abstract syntax in a few ways: \verb+pi x\+ and \verb+sigma x\+ are used in place of $\forall x.$ and $\exists x.$; $P \Rightarrow Q$ can be written as \verb+P => Q+ or as \verb+Q :- P+; conjunction is written using commas; $\lambda x.$ is written \verb+x\+. A program is just a list of clauses terminated by dots. All free uppercase names in a clause are implicitly universally quantified around the clause. The user writes a query to be resolved against a program. All free uppercase names in the query are also implicitly existentially quantified. The semantics of the pair program/clause is a new query obtained as the implication between the $n$-ary conjunction of all the clauses and the user provided query.
% 
% $\lambda x.A$ and $\forall x.Q$ bind $x$ respectively in $A$ and $Q$;
% $\forall X.P$ and $\exists X.Q$ bind $X$ respectively in $P$ and $Q$.
% Formulas are identified up to $\alpha$-equality as usual.
% 
% \paragraph{Definition and expressivity of the \frag}
% The De Bruijn level of a name $x$ in an expression is the number of binders to be crossed when traversing the expression from the root towards the leaves before finding the binder for $x$~\cite{debruijnlevel}. For example, in the following expression every variable $x_i$ has level $i$: $\forall x_0.\exists X^1.\lambda x_1.\lambda x_2. \exists X^3. p~(X^1~x^1) X^3$. Every expression can be
% $\alpha$-converted so that every name is renamed to $x_i$ where $i$ is its level. In the rest of the discussion we assume every term to be renamed in this way. Similarly, we can rename every variable $X$ to $X^j$ where $j$ is the smallest number such that $X^j$ has visibility of all names smaller than $j$. In the example above, $X^1$ has visibility only of $x^0$, and $X^3$ has visibility of $\{x^0,x^1,x^2\}$.
% 
% An expression is in the \frag{} iff every occurrence of an existentially quantified $X^j$ occurs applied to $x_j \ldots x_{j+k-1}$ for $k >= 0$.
% 
% Examples: $X^2~x_2~x_3$ and $X^2$ are in the fragment; $X^2~x_3$ and $X^2~x_3~x_2$ are not.

\begin{myprop}[Decidability of HO unification]
Being the RFF included in the pattern-fragment, higher order unification is
decidable for the RFF.
\end{myprop}
%In other words, each problem of the form $X^j~x_j\ldots x_{j+k-1} \equiv t$
%admits a MGU.

The most interesting property of the RFF, which also justify its name, is the
following one. 

\begin{myprop}[Constant time head $\beta$-reduction]
Let $\sigma$ be a \emph{valid} substitution for existentially quantified variables.  Then the head normal form of $(X^j~x_j \ldots x_{j+k-1}) \sigma$
can be computed in constant time.
\end{myprop}

A valid substitution assigns to $X^j$ a term $t$ of the right type (as in simply
type lambda calculus) and that has all free variables visible by $X^j$ (all
$x_i$ are such that $i < j$).  
Let $X^j \sigma = \lambda x_j. \ldots \lambda x_{j+n}.t$. Then
\begin{equation}\label{deffrag}(X^j~x_j \ldots x_{j+k-1}) \sigma
 = \left\{ \begin{array}{ll}
t~x_{j+n+1} \ldots x_{j+k-1} & \mbox{if $n+1 < k$} \\
\lambda x_{j+k}. \ldots \lambda x_{j+n}.t & \mbox{otherwise}
      \end{array} \right.\end{equation}
Thanks to property \textbf{DBL2}, Equation~\ref{deffrag} is
\emph{syntactical}: no lifting of $t$ is required.
Hence the $\beta$-reductions triggered by the substitution of $X^j$ take
constant time.

\begin{myprop}[Constant time unification]
A unification problem of the form  $X^j~x_j\ldots x_{j+k-1} \equiv t$
can be solved in constant time.
\end{myprop}

The unification problem $X^j~x_j\ldots x_{j+k-1} \equiv t$ can always be
rewritten as two simpler problems: $X^j \equiv \lambda x_j. \ldots \lambda x_{j+k-1}. Y^{j+k}$ and $Y^{j+k} \equiv t$ for a fresh $Y$.
The former is a trivial assignment that requires no check.
The latter can be implemented in constant time if: no occur-check is needed
for $X$ and if one caches in the terms the level of the highest free variable.
Avoiding useless occur-check is a typical optimization of the Warren Abstract
Machine (WAM), e.g. when $X$ occurs linearly in the head of a clause.
Caching in the terms the maximum free level is economical in terms of space
(just one integer) and is something one typically pre-computes on the input
term in linear time.  With such maximum level $l$ at hand the problem $Y^{j+k}
\equiv t$ can be decided by simply comparing $j+k$ with $l$.
These properties enable us to implement the operational semantics of \verb+pi+
in constant time for terms in the RFF.

We detail an example.
The first column gathers the fresh constants and extra clauses. The
second one shows the current goal(s) and the program clause that is
used to back chain.

\begin{center}
\small
\begin{tabular}{c|l}
Context & Goals and refreshed program clause \\\hline
& \verb+of (lam x+$_0$\verb+\lam x+$_1$\verb+\app x+$_0$\verb+ x+$_1$\verb+) T+$^0$  \\
& \verb+of (lam F+$^0$\verb+) (arr A+$^0$\verb+ B+$^0$\verb+) :- pi x+$_0$\verb+\ of x+$_0$\verb+ A+$^0$\verb+ => of (F+$^0$\verb+ x+$_0$\verb+) B+$^0$ \\\hline
\verb+ x+$_0$;\verb+(of x+$_0$\verb+ A+$^0$\verb+)+ & \verb+of (lam x+$_1$\verb+\app x+$_0$\verb+ x+$_1$\verb+) B+$^0$  \\
& \verb+of (lam G+$^1$\verb+) (arr C+$^1$\verb+ D+$^1$\verb+) :- pi x+$_1$\verb+\ of x+$_1$\verb+ C+$^1$\verb+ => of (G+$^1$\verb+ x+$_1$\verb+) D+$^1$\verb++  \\\hline
\verb+ x+$_0$;\verb+(of x+$_0$\verb+ A+$^0$\verb+)+ & \verb+of (app x+$_0$\verb+ x+$_1$\verb+) D+$^0$  \\
\verb+ x+$_1$;\verb+(of x+$_1$\verb+ C+$^0$\verb+)+ & \verb+of (app M+$^2$\verb+ N+$^2$\verb+) S+$^2$\verb+ :- of M+$^2$\verb+ (arr R+$^2$\verb+ S+$^2$\verb+), of N+$^2$\verb+ R+$^2$  \\\hline
\verb+ x+$_0$;\verb+(of x+$_0$\verb+ A+$^0$\verb+)+ & \verb+of x+$_0$\verb+ (arr R+$^2$\verb+ S+$^0$\verb+), of x+$_1$\verb+ R+$^2$  \\
\verb+ x+$_1$;\verb+(of x+$_1$\verb+ C+$^0$\verb+)+ & \verb+of x+$_0$\verb+ A+$^0$\hspace{4pt}\verb+        , of x+$_1$\verb+ C+$^0$ \\\hline
\end{tabular}
\end{center}

After the first step we obtain
\verb+F+$^0$\verb+:= x+$_0$\verb+\lam x+$_1$\verb+\app x+$_0$\verb+ x+$_1$;
\verb+T+$^0$\verb+:= arr A+$^0$\verb+ B+$^0$; the extra clause
about \verb+x+$_0$ in the context and a new subgoal.
Note that the redex \verb+(F+$^0$\verb+ x+$_0$\verb+)+ is in the RFF and
thanks to Equation~\ref{deffrag} head normalizes in constant time
to \verb+(lam x+$_1$\verb+\app x+$_0$\verb+ x+$_1$\verb+)+.
The same phenomenon arises in the second step,
where we obtain \verb+G+$^1$\verb+:= x+$_1$\verb+\app x+$_0$\verb+ x+$_1$
and we generate the redex \verb+(G+$^1$\verb+ x+$_1$\verb+)+.
Unification variables are refreshed in the context under with
the clause is used, e.g. \verb+C+ is placed at level 1 initially,
but in consequence to a unification step they may be \emph{pruned}
when occurring in a term assigned to a lower level unification
variable. Example:  unifying 
\verb+B+$^0$ with \verb+(arr C+$^1$\verb+ D+$^1$\verb+)+ prunes
\verb+C+ and \verb+D+ to level 0.

The choice of using DBL for bound variables is both an advantage and a
complication here.
Clauses containing no bound variables, like
\verb+(of x+$_0$\verb+ A+$^0$\verb+)+, require no processing thanks to
\textbf{DBL1}: they can be indexed as they are, since the name \verb+x+$_0$
is stable.
The drawback is that clauses with bound variables, like the one 
used in the first two back chains, need to be lifted: the first time the
bound variable is named \verb+x+$_0$,
while the second time \verb+x+$_1$.
Luckily, this renaming, thanks property \textbf{DBL1}
can be performed in constant time using the very same machinery one uses to
refresh the unification variables.
E.g. when the WAM unifies the head of a clauses it assigns 
fresh stack cells: the clause is not really refreshed and the stack
pointer is simply incremented.  One can represent the locally bound variable as
an extra unification variable, and initialize, when \verb+pi+ is crossed, the
corresponding stack cell to the first \verb+x+$_i$ free in the context.

\paragraph{Stability of the RFF.}
Unlike \Ll{}, the RFF is not stable under \lp{} resolution:
a clause that contains only terms in the RFF
may generate terms
outside the fragment
because
of the lifting phenomenon explained the previous paragraph.\marginpar{resurrect Claudio's example?}
Therefore an implementation must handle both terms in the RFF, with their
efficient computation rules, and terms outside the fragment. Our limited
experience so far, however, is that several programs initially written in the
fragment remains in the fragment during computation, or they can be slightly
modified to achieve that property.


% The term $X^j~x_j\ldots x_{j+k-1}$ can always be rewritten as
% $Y^{j+k}$ instantiating $X^j$ with $\lambda x_j. \ldots \lambda x_{j+k-1}. X^{j+k}$ for a fresh existentially quantified variable $X^{j+k}$. Therefore every
% unification problem in the \frag{} can be reduced in linear time in $k$ to a
% problem of the kind $X^{j+k} \equiv \mathcal{E}$, that is immediately solved
% by instantiating $X^{j+k}$ with $\mathcal{E}$ if the free names of $E$ are
% known to be a subset of $\{x_0,\ldots,x_{j+k-1}\}$.
% 
% Note that, once all $\beta$-redex that occur syntactically in the term are
% fired at compile time, or if explicitly written $\beta$-redexes are forbidden,
% then every $\beta$-redex in the computation is triggered by the substitution
% of an existentially quantified variable. Therefore, if the program is and
% remains in the RFF, then all $\beta$-reductions can be implemented in
% constant time.
% 
% Finally, another benefit of working with De Bruijn levels is that the
% $\forall$-introduction rule can also be implemented in constant time as well
% because there is no need to replace the bound variable with a fresh name.
% The price to pay is an additional complexity in the backchain rule: when
% a clause is selected, the clause needs to be $\alpha$-converted --- lifted
% in De Bruijn indexes/levels terminology --- to move it in the current scope.
% For example, if the query is \verb+ pi x_0\ r x_0+ and \verb+r X :- pi x_0\ pi x_1\ p x_0 x_1 X+ is a clause, after two inferences the new query becomes
% \verb+pi x_1\ pi x_2\ p x_1 x_2 x_0+ because the introduction rule for
% \verb+pi+ has already fixed the name \verb+x_0+.
% 
% \paragraph{Instability of the \frag}
% Unlike the pattern fragment, the \frag{} is unstable for computation.
% I.e. there are simple examples where the program and the initial query are
% written in the fragment, but during execution we generate terms outside the
% fragment (but still in the pattern fragment). This is a consequence of the
% need for lifting clauses during backchaining.
% Consider example~5 in Table~\ref{example4}.
% After two inference steps the term \verb+(x1\ X^0 x0 x1)+, that is in
% the fragment and assigned to \verb+Y+, is moved under \verb+pi x0+ becoming
% \verb+r (x2\ X^0 x0 x2)+. The latter is no longer in the fragment.
% 
% Therefore, and contrarily to what happens with the pattern fragment, an
% implementation must handle both terms in the fragment, with their efficient
% computation rules, and terms outside the fragment. Our limited experience so
% far, however, is that several programs initially written in the fragment
% remains in the fragment during computation, or they can be slightly modified
% to achieve that property.
% 
% \section{\elpi: an Embedded \lp{} Interpreter.}\label{sec:elpi}
% The current version of \elpi, together with the benchmarks presented in
% the next section, can be downloaded at \CSC{\url{http://xxxxx}}. The intepreter
% is entirely written in OCaml and it is \CSC{XXX} lines long. It augments the language presented in the paper with Prolog-style cuts and a few custom predicates for printing. The name is due to the fact that we eventually plan to augment the language with constraints and then embed it in the implementation of interactive theorem provers like Coq, that are often written in OCaml.
% 
% We carefully wrote the code so that unrechable terms in \lp{} are encoded by unreachable terms in OCaml. Therefore we inherit garbage collection from the OCaml runtime. The algebraic datatype used to encode terms has different constructors for occurrences of terms in the \frag, and for occurrences outside the fragment. In particular an occurrence $X^j x_j \ldots x_{j+k-1}$ is simply represented as a triple $\langle r,j,k\rangle$ where $r$ is a reference to the term
% $X$ is instantiated to, or to a dummy term if $X$ is still unbound. Finally, we carefully cherry picked some design decisions from the WAM, implementing our own variations in several places. The variations have been made necessary to decrease the pressure on the garbage collector of OCaml, that is responsible for a significant percentage of the running time. For example, the heap and the stack are not organized as large arrays of mostly unused cells because otherwise the garbage collector needs to traverse the arrays at every minor and major collection. Similarly, we employ a mix of mutable and persistent data structures, avoiding mutable structures for data that is likely to survive, in order to avoid the penalty of the write barrier on data in the old generation~\cite{???}.
% Further details will be reported in a future work due to lack of space.

\section{Assessment and conclusions}\label{sec:benchmarks}

We asses the performances of \elpi{} on a set of synthetic benchmarks and
a real application.  Synthetic benchmarks are divided into three groups:
first order programs from the Aquarius test suite (the
crypto-multiplication, $\mu$-puzzle,
generalized eight queens problem and the Einstein's zebra puzzle);
higher order programs falling in the RFF fragment; and an higher
order program falling outside RFF taken from the test suite of
\tedius{} normalizing expressions in the SKI calculus.

The programs in the RFF are respectively type checking lambda terms using
the \verb+of+ program of Example~\ref{example1} and reducing expressions like
$5^5$ using Church numerals using a call by value strategy. \verb+lambda3+ was
specifically conceived to measure the cost of moving under binders.

\begin{center}
  \scriptsize 
  \begin{tabular}{|p{1.5cm}||c|r||c|r||c|c|}
    \hline
      \multicolumn{1}{|c||}{Test} &
      \multicolumn{2}{|c||}{ELPI} &
      \multicolumn{2}{|c||}{\tedius{}} &
      \multicolumn{2}{|c|}{ELPI/\tedius{}} \\
    \hline
    &  time (s)     & space (Kb)  & time (s) & space (Kb) &  time & space \\
    
    \hline
    \hline
    crypto-mult &  3.46 & 25,828  & 6.59 & 18,048 &  0.52 & 1.43 \\
    \hline    
    $\mu$-puzzle &  1.83 & 5,716 &  3.62 & 50,076 &  0.50 & 0.11 \\
    \hline
    queens &  1.33  & 108,140 &  2.02 & 69,968 &  0.65 & 1.54 \\
    \hline    
    zebra &  0.85 & 6,944 &  1.89 & 8,412 &  0.44 & 0.82 \\
    \hline     
    \hline
    typeof &  0.30 & 6,080 &  5.64 & 239,892 &  0.05 & 0.02 \\
    \hline
    reduce\_cbv &  0.15 & 7,288 &   11.17 & 80,736  & 0.01 & 0.08 \\
    \hline
    \hline
    SKI &  1.30 & 21,668 &  2.68 & 8,896  & 0.48 & 2.43 \\
    \hline
    
  \end{tabular}
\end{center}

The table shows that \elpi{} shines on programs in the RFF, and compares well
outside it.  It is hard to understand why \elpi{} is faster than \tedius{}
outside RFF, since all the optimizations we used are inspired by the
literature about the WAM, on which \tedius{} is also based.  Our guess is that
the OCaml garbage collector, well known for its efficiency, is responsible for
the difference. Especially because we heavily optimized our code to lower the
pressure on it to let it perform at the best of its
capabilities.

\subsection{A Relevant Test Case: the ``Grundlagen'' Verified}
\label{grundlagen}

As a relevant test case for \elpi,
we implemented in the \frag{}
a validator for the latest version of the formal system $\lambda\delta$,
and used this validator to verify the
``Grundlagen'' \cite{Jut79} translated in a Pure Type System
\cite{Brn92}.

The formal system $\lambda\delta$ \cite{lambdadeltaJ1},
improved in \cite{lambdadeltaJ2a,lambdadeltaJ3a},
is a framework that embeds
some former typed $\lambda$-calculi including 
$Aut-QE$, the Automath dialect in which the ``Grundlagen'' was
originally written, and Pure Type Systems like $\lambda C$.

%FG: comment this paragraph in case of need
Current verification algorithms for typed systems follow
a well-established pattern prescribing a reduction
machine to compute weak head normal forms, a comparator to assert
convertibility by levels, and a checker responsible for type inference.
The verification algorithm for $\lambda\delta$ 
deviates slightly from this pattern in that
type checking is replaced by validation, and in that
type inference is delegated to the (extended) reduction machine. 
The performance benefits of this approach are documented in
\cite{lambdadeltaJ3a}.
We wish to recall that type checking a term means
asserting that this term has a specified type, whereas
validating a term means asserting that this term has 
some unspecified type.

A validator for $\lambda\delta$, named Helena,
has been implemented in Caml,
and our \lp{} implementation follows it closely.
Nevertheless, the \lp{} code is much simpler that the
corresponding Caml code, and consists of just 52 clauses.

The translated ``Grundlagen'' is a theory comprising 
32 declarations and 6879 definitions, for a total of 6911 items.
Each item is a term to be verified, written in the raw syntax of
$\lambda C$ with constants type casts. The sort $\square$ never
appears explicitly and the sorts \emph{set} or \emph{prop}
are used in place of the sort $\ast$.

The term in each definition is a type cast in a 
context of $\lambda$-abstractions corresponding to the ``block openers''
of $Aut-QE$.
On the other hand, the term in each declaration (that is, a type) is
given in a context of $\Pi$-abstractions.

Overall, the tree representation of these terms consists of
754579 nodes.
Due to this huge amount of data, the validation with Teyjus is work in
progress.
On the other hand, we can present the translated ``Grundlagen'' to Coq.

\cite{lambdadeltaJ3a}.

\begin{center}
\begin{tabular}{|l|c|c|c|}
\hline
\multicolumn{4}{|c|}{User time range (s) for 31 runs}\\
\hline
Execution   & Helena              & \elpi               & Coq                 \\
\hline
compiled    & from 01.17 to 01.20 & not applicable      & from 24.26 to 24.43 \\
\hline
interpreted & from 08.74 to 08.78 & from 27.52 to 27.82 & from 94.18 to 95.78 \\
\hline
\end{tabular}
\end{center}

 
\bibliographystyle{plain}
\bibliography{reference}

\end{document} 
