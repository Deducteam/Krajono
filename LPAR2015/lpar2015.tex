% LLNCStmpl.tex
% Template file to use for LLNCS papers prepared in LaTeX
%websites for more information: http://www.springer.com
%http://www.springer.com/lncs

\documentclass{llncs}
%Use this line instead if you want to use running heads (i.e. headers on each page):
%\documentclass[runningheads]{llncs}
\usepackage{amssymb}
\usepackage{graphicx}
\usepackage{url}
\usepackage{multirow}

\begin{document}
\title{Title }

%If you're using runningheads you can add an abreviated title for the running head on odd pages using the following
%\titlerunning{abreviated title goes here}
%and an alternative title for the table of contents:
%\toctitle{table of contents title}

%\subtitle{Subtitle Goes Here}

%For a single author
%\author{Author Name}

%For multiple authors:

\author{Authors}


%If using runnningheads you can abbreviate the author name on even pages:
%\authorrunning{abbreviated author name}
%and you can change the author name in the table of contents
%\tocauthor{enhanced author name}

%For a single institute
\institute{Department of Computer Science,
University of Bologna\\ 
Mura Anteo Zamboni 7, 40127 Bologna, Italy \\
\email{emails@cs.unibo.it}
\and INRIA Sophia Antipolis\\ 2004 route des Lucioles - BP 93, 06902 Sophia Antipolis Cedex, France
\email{emails@inria.fr}}

% If authors are from different institutes
%\institute{First Institute Name \email{email address} \and Second Institute Name\thanks{Thank you to...} \email{email address}}

%to remove your email just remove '\email{email address}'
% you can also remove the thanks footnote by removing '\thanks{Thank you to...}'


\maketitle

\begin{abstract}
abstract here
\end{abstract}


\section{Introduction}
section \cite{ref}. 

%  \begin{tabular}{ | l | l | l | p{5cm} |}




\begin{center}

 \begin{table}
  \begin{tabular}{|p{1.1cm}||p{1.1cm}|p{1.1cm}|p{1.1cm}||p{1.1cm}|p{1.1cm}|p{1.1cm}||p{1.1cm}|p{1.1cm}|p{1.1cm}|}
    \hline
      \multicolumn{1}{|c||}{Test} &
      \multicolumn{3}{|c||}{Elpi} &
      \multicolumn{3}{|c||}{Teyjus} &
      \multicolumn{3}{|c|}{Ratio(Elpi/Teyjus)} \\
    \hline
      & memory & clock time & user time & memory & clock time & user time & memory & clock time & user time \\
    \hline
    XXX & XXX & ZZZ & AAA & BBB & CCC & DDD & CCC & EEE & FFF \\
    \hline
    XXX & XXX & ZZZ & AAA & BBB & CCC & DDD & CCC & EEE & FFF \\
    \hline
    XXX & XXX & ZZZ & AAA & BBB & CCC & DDD & CCC & EEE & FFF \\
    \hline
  \end{tabular}
 \end{table}

 \end{center}


\bibliographystyle{plain}
\bibliography{reference}

\end{document} 